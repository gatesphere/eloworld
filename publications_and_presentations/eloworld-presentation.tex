% Eloworld
% Jacob Peck
% COG 499 - Models of Complexity - Vampola
% 20120312

\documentclass[12pt]{beamer}
\usetheme{Warsaw}
\usepackage{bera}
%\usepackage[all]{xy}

\begin{document}

\title{Eloworld - a game-theoretic simulation of social grouping}
\author{Jacob M. Peck}
\date{March 23, 2012}
\maketitle

\section{Preliminaries}
\subsection{Some terminology}
\begin{frame}{Some terminology}
  \begin{description}
    \item[Intelligent agents] are ``autonomous entit[ies] which observe through sensors and act upon an environment using actuators ... and direct [their] activity towards achieving goals.''
    \item[Social capital] is ``the value of social relations and the role of cooperation and confidence to get collective or economic results.''
    \item[Social grouping] is ``two or more humans who interact with one another, share similar characteristics and collectively have a sense of unity.''
  \end{description}
\end{frame}

\subsection{What is Eloworld?}
\begin{frame}{What is Eloworld?}
  \begin{itemize}
    \item A game-theoretic simulation of social grouping 
    \item An experiment in computational modelling 
    \item An exploration of swarm based intelligence
    \item An examination of the ELO rating system
  \end{itemize}
\end{frame}

\section{Architecture}
\subsection{The World}
\begin{frame}{The World}
  \begin{itemize}
    \item A two-dimensional Cartesian plane
    \item Finite in both dimensions
    \item Non-toroidal
    \item Obstacle and collision free
  \end{itemize}
\end{frame}

\subsection{The Game}
\begin{frame}{The Game}
  \begin{itemize}
    \item Zero-sum (qualitatively)
    \item Glorified dice rolls, with a bias
    \begin{itemize}
      \item Each player has a true skill in the range (0 .. 1)
      \item Each player then makes a roll in the range (true skill .. 1)
      \item The players add their true skill to their roll, and compare 
      \begin{itemize}
        \item If difference of scores <= 0.1, draw (each earns .5 points)
        \item Else, higher score gains 1 point, lower score gains 0 points.
      \end{itemize}
    \end{itemize}
  \end{itemize}
\end{frame}

\subsection{The ELO Rating System}
\begin{frame}{The ELO Rating System}
  \begin{itemize}
    \item A simple numeric rating of so-called ``proven skill''
    \item Devised by Arpad Elo for competitive Chess
  \end{itemize}
  \begin{block}{Mathematical details}
    Expected score for player $A$ facing player $B$ is (0 .. 1):
    \begin{center}
    $E_A = \frac 1 {1 + 10^{(R_B - R_A)/400}}$
    \end{center}
    Adjusted rating for player $A$ after a game with player $B$:
    \begin{center}
    $R_A^\prime = R_A + K(S_A - E_A)$
    \end{center}
    Where $K$ is a sliding constant based upon the player's current ELO rating.
  \end{block}
\end{frame}

\subsection{The Agents}
\begin{frame}{The Agents}
  \begin{itemize}
    \item Consist of various properties:
    \begin{itemize}
      \item position (x,y)
      \item current ELO score
      \begin{itemize}
        \item letter-grade ranking
        \item current $K$-value
        \item differential
      \end{itemize}
      \item true skill (0..1)
      \item whim (0..1)
      \item number of games played
      \item list of recently played opponents
    \end{itemize}
  \end{itemize}
\end{frame}

\begin{frame}{The Agents}
  \begin{itemize}
    \item Simple behavior:
    \begin{itemize}
      \item Find target
      \begin{itemize}
        \item Look at every agent's ELO
        \item Find the highest ranked player within differential
        \item Reject if in the recently played list
      \end{itemize}
      \item Move towards target
      \begin{itemize}
        \item If no valid target, move in a random direction
      \end{itemize}
      \item If in same location as target, challenge to a game
      \begin{itemize}
        \item If within target's differential, a game is played
        \item Else, if target's whim roll is high, play a game
        \item Else, target rejects, no game is played
      \end{itemize}
    \end{itemize}
  \end{itemize}
\end{frame}

\section{Application}
\subsection{Results}
\begin{frame}{Results}
  \begin{itemize}
    \item Emergence
    \begin{itemize}
      \item Agents actively strive to increase and protect their ELO rating
      \item Agents vary game playing partners
    \end{itemize}
    \item Convergence
    \begin{itemize}
      \item Agents tend to converge around the center of the world
      \item Such convergence leads to one large social group
    \end{itemize}
    \item Social mobility
    \begin{itemize}
      \item Agents are able to move through social classes by way of changing ELO ratings
      \item Agents act in a way that mirrors their social standing
    \end{itemize}
  \end{itemize}
\end{frame}

\subsection{Demo time?}
\begin{frame}{Demo time?}
  \begin{itemize}
    \item One second while I set things up...
  \end{itemize}
\end{frame}

\section{End}
\subsection{Sources}
\begin{frame}{Sources}
  \begin{thebibliography}{19}
    \bibitem{1} \url{https://github.com/gatesphere/eloworld}
    \bibitem{2} \url{http://en.wikipedia.org/wiki/Intelligent_agent}
    \bibitem{3} \url{http://en.wikipedia.org/wiki/Social_capital}
    \bibitem{4} \url{http://en.wikipedia.org/wiki/Social_group}
    \bibitem{5} \url{http://en.wikipedia.org/wiki/Elo_rating_system}
  \end{thebibliography}
\end{frame}

\subsection{Any questions?}
\begin{frame}{Any questions?}
  \begin{itemize}
    \item Feel free to ask.
  \end{itemize}
\end{frame}

\end{document}
